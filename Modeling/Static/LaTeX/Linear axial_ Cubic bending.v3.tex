\documentclass[preprint,12pt]{elsarticle}
\usepackage{amsmath,amssymb}
\usepackage{geometry}
\geometry{margin=1in}

\begin{document}
	
	\begin{frontmatter}
		
		\title{Extended Euler--Bernoulli Beam with Axial DOF:\\
			Strong, Weak, and Discrete Formulations}
		
	
	\end{frontmatter}
	
	\section*{Nomenclature}
	\begin{tabular}{p{2cm} p{1cm} p{11cm}}
		$E$           & [Pa]    & Young’s modulus (steel: $2.1\times10^{11}$).\\
		$A$           & [m$^2$] & Cross‐sectional area (e.g.\ $0.01$).\\
		$I$           & [m$^4$] & Second moment of area (e.g.\ $8.333\times10^{-6}$).\\
		$\rho$        & [kg/m$^3$]& Mass density (e.g.\ $7850$).\\
		$L$           & [m]     & Total beam length (e.g.\ $2$).\\
		$L_e$         & [m]     & Element length: $L_e=L/n_{\mathrm{elem}}$.\\
		$x$           & [m]     & Spatial coordinate.\\
		$t$           & [s]     & Time coordinate.\\
		$u(x,t)$      & [m]     & Axial displacement.\\
		$w(x,t)$      & [m]     & Transverse displacement.\\
		$\delta u,\;\delta w$ & – & Virtual (test) displacements.\\
		$k_e$         & –       & Full element stiffness matrix (6×6).\\
		$m_e$         & –       & Full element mass matrix (6×6).\\
	\end{tabular}
	
	\section{Governing Equations}
	The classical Euler–Bernoulli beam equations for transverse deflection \(w(x,t)\) and axial displacement \(u(x,t)\) are
	\[
	EI\,\frac{\partial^4w}{\partial x^4}(x,t)
	\;+\;\rho A\,\frac{\partial^2w}{\partial t^2}(x,t)
	\;=\;q(x,t),
	\tag{1}
	\]
	
	With an added axial degree of freedom, the beam’s coupled strong form becomes
	\[
	\begin{cases}
		\displaystyle
		\rho A\,\frac{\partial^2 u}{\partial t^2}
		\;-\;
		\frac{d}{dx}\bigl(EA\,u'\bigr)
		\;=\;q(x,t),
		\\[8pt]
		\displaystyle
		\rho A\,\frac{\partial^2 w}{\partial t^2}
		\;-\;
		\frac{d^2}{dx^2}\bigl(EI\,w''\bigr)
		\;=\;q(x,t).
	\end{cases}
	\tag{3}
	\]
	Here the first equation governs axial motion and the second is the transverse beam equation including the external load \(q\).
	
	\section{Weak Form}
	To reduce the continuity requirements on trial and test spaces and to impose natural boundary conditions, we derive the variational form by multiplying each strong equation by its virtual displacement and integrating by parts.
	
	\subsection{Axial}
	Multiply Eq.~(3)$_1$ by \(\delta u\), integrate by parts:
	\[
	\int_0^L EA\,u'\,\delta u'\,dx - \bigl[EA\,u'\,\delta u\bigr]_0^L = 0
	\;\Longrightarrow\;
	\int_0^L EA\,u'\,\delta u'\,dx = 0.
	\tag{2}
	\]
	This weak form requires only \(u\in H^1\) (i.e.\ \(C^0\)) continuity and enforces zero virtual displacement at fixed ends.
	
	\subsection{Bending}
	Multiply Eq.~(3)$_2$ by \(\delta w\), integrate by parts twice:
	\[
	\int_0^L EI\,w''\,\delta w''\,dx - \bigl[EI\,w''\,\delta w' - (EI\,w'')'\,\delta w\bigr]_0^L = 0
	\;\Longrightarrow\;
	\int_0^L EI\,w''\,\delta w''\,dx = 0.
	\tag{3}
	\]
	This requires \(w\in H^2\) (i.e.\ \(C^1\)) continuity and naturally imposes zero virtual deflection and slope at fixed ends.
	
	\section{Finite‐Element Interpolations}
	\subsection{Axial (linear)}
	We approximate \(u(x)\) over each element of length \(L_e\) by
	\[
	N^{(u)}(x)
	= \begin{pmatrix}1-\tfrac{x}{L_e}\\[4pt]\tfrac{x}{L_e}\end{pmatrix},
	\qquad
	u(x)=\bigl[N^{(u)}(x)\bigr]^T(u_1,u_2)^T.
	\tag{4}
	\]
	Because \(dN^{(u)}/dx=[-1/L_e,\;1/L_e]^T\) is constant, the axial strain is uniform, simplifying the stiffness integral.
	
	\subsection{Transverse (Hermite cubic)}
	To capture bending curvature with \(C^1\) continuity, we use four cubic polynomials:
	\[
	\begin{aligned}
		N_1(x)&=1-3(\tfrac{x}{L_e})^2+2(\tfrac{x}{L_e})^3,\\
		N_2(x)&=x[1-2(\tfrac{x}{L_e})+(\tfrac{x}{L_e})^2],\\
		N_3(x)&=3(\tfrac{x}{L_e})^2-2(\tfrac{x}{L_e})^3,\\
		N_4(x)&=x[-(\tfrac{x}{L_e})+(\tfrac{x}{L_e})^2].
	\end{aligned}
	\]
	Each \(N_i(x)\) is cubic in \(x\).  For integration, map \(x\to\xi=x/L_e\in[0,1]\):
	\[
	\begin{aligned}
		N_1(\xi)&=1-3\xi^2+2\xi^3,\quad
		N_2(\xi)=L_e\,\xi(1-2\xi+\xi^2),\\
		N_3(\xi)&=3\xi^2-2\xi^3,\quad
		N_4(\xi)=L_e\,\xi(-\xi+\xi^2).
	\end{aligned}
	\tag{5}
	\]
	Then \(w(x)=N^{(w)}(\xi)\,(w_1,\theta_1,w_2,\theta_2)^T\).
	
	\section{Element Stiffness Matrices}
	Substituting the above interpolations into the weak forms (2)–(3) and evaluating the resulting integrals yields:
	
	\subsection{Axial}
	By inserting \(u(x)\) from (4) into (2) and noting the constant derivative,
	\[
	k_e^{(\mathrm{axial})}
	= \int_0^{L_e}EA\,\frac{dN^{(u)}}{dx}\frac{dN^{(u)}}{dx}^Tdx
	= \frac{EA}{L_e}
	\begin{pmatrix}1 & -1\\ -1 & 1\end{pmatrix}.
	\tag{6}
	\]
	This 2×2 block arises directly from the axial strain energy.
	
	\subsection{Bending}
	From the bending weak form (3), define the curvature–shape vector in $\xi$:
	\[
	B(\xi)
	= \frac{d^2N^{(w)}}{dx^2}
	= \frac{1}{L_e^2}[6\xi-6,\;3\xi-4,\;-6\xi+6,\;3\xi-2],
	\tag{7a}
	\]
	then from (3):
	\[
	k_e^{(\mathrm{bending})}
	= EI\int_0^{L_e}B^T B\,dx
	= \frac{EI}{L_e^3}
	\begin{pmatrix}
		12 & 6L_e & -12 & 6L_e\\
		6L_e & 4L_e^2 & -6L_e & 2L_e^2\\
		-12 & -6L_e & 12 & -6L_e\\
		6L_e & 2L_e^2 & -6L_e & 4L_e^2
	\end{pmatrix}.
	\tag{7b}
	\]
	This 4×4 block represents the bending stiffness derived from curvature energy \(\tfrac12\int EI\,w''^2dx\).
	
	\subsection{Assembled stiffness}
	Finally, the full 6×6 element stiffness matrix is
	\[
	k_e
	= \begin{pmatrix}
		k_e^{(\mathrm{axial})} & \mathbf{0}\\
		\mathbf{0}            & k_e^{(\mathrm{bending})}
	\end{pmatrix}.
	\tag{8}
	\]
	
	\section{Element Mass Matrices}
	Analogously, from the kinetic energies we obtain:
	
	\subsection{Axial}
	Using \(\tfrac12\int\rho A\,\dot u^2dx\) and shape (4):
	\[
	m_e^{(\mathrm{axial})}
	= \int_0^{L_e}\rho A\,N^{(u)}N^{(u)T}dx
	= \frac{\rho A L_e}{6}
	\begin{pmatrix}2 & 1\\ 1 & 2\end{pmatrix}.
	\tag{9}
	\]
	
	\subsection{Bending}
	Using \(\tfrac12\int\rho A\,\dot w^2dx\) and Hermite shapes:
	\[
	m_e^{(\mathrm{bending})}
	= \rho A L_e\int_0^1N^{(w)}(\xi)^T N^{(w)}(\xi)d\xi
	= \frac{\rho A L_e}{420}
	\begin{pmatrix}
		156 & 22L_e & 54 & -13L_e\\
		22L_e & 4L_e^2 & 13L_e & -3L_e^2\\
		54 & 13L_e & 156 & -22L_e\\
		-13L_e & -3L_e^2 & -22L_e & 4L_e^2
	\end{pmatrix}.
	\tag{10}
	\]
	
	\subsection{Assembled mass}
	The full 6×6 mass matrix is
	\[
	m_e
	= \begin{pmatrix}
		m_e^{(\mathrm{axial})} & \mathbf{0}\\
		\mathbf{0}            & m_e^{(\mathrm{bending})}
	\end{pmatrix}.
	\tag{11}
	\]
	
	\bibliographystyle{elsarticle-num}
	\bibliography{references}
	
\end{document}
