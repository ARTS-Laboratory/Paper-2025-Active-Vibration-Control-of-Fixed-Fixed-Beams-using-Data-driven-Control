\documentclass[preprint,12pt]{elsarticle}
\usepackage{amsmath,amssymb}
\usepackage{geometry}
\geometry{margin=1in}

\begin{document}
	
	\begin{frontmatter}
		\title{Extended Euler--Bernoulli Beam with Axial DOF:\\
			Strong, Weak, and Discrete Formulations}
	\end{frontmatter}
	
	\section*{Nomenclature}
	\begin{tabular}{p{2cm} p{1cm} p{11cm}}
		$E$           & [Pa]    & Young’s modulus (steel: $2.1\times10^{11}$).\\
		$A$           & [m$^2$] & Cross‐sectional area (e.g. $0.01$).\\
		$I$           & [m$^4$] & Second moment of area (e.g. $8.333\times10^{-6}$).\\
		$\rho$        & [kg/m$^3$]& Mass density (e.g. $7850$).\\
		$L$           & [m]     & Total beam length (e.g. $2$).\\
		$L_e$         & [m]     & Element length: $L_e=L/n_{\mathrm{elem}}$.\\
		$x$           & [m]     & Spatial coordinate.\\
		$t$           & [s]     & Time coordinate.\\
		$u(x,t)$      & [m]     & Axial displacement.\\
		$w(x,t)$      & [m]     & Transverse displacement.\\
		$\delta u,\;\delta w$ & – & Virtual (test) displacements.\\
		$k_e$         & –       & Full element stiffness matrix (6×6).\\
		$m_e$         & –       & Full element mass matrix (6×6).\\
	\end{tabular}
	
	\section{Governing Equations}
	The classical Euler–Bernoulli beam equation for transverse deflection \(w(x,t)\) is
	\begin{equation}\label{eq:transverse_classical}
		EI\,w''''(x,t)
		+\rho A\,\ddot w(x,t)
		= q(x,t).
	\end{equation}
	
	With an added axial degree of freedom, the beam’s strong form separates into two equations:
	\begin{equation}\label{eq:axial_strong}
		\rho A\,\ddot u(x,t)
		- \bigl(EA\,u'(x,t)\bigr)'
		= q(x,t),
	\end{equation}
	\begin{equation}\label{eq:transverse_strong}
		\rho A\,\ddot w(x,t)
		- \bigl(EI\,w''(x,t)\bigr)''
		= q(x,t).
	\end{equation}
	
	\section{Weak Form}
	To reduce the continuity requirements on trial and test spaces and to impose natural boundary conditions, we derive the variational form by multiplying each strong equation by its virtual displacement and integrating by parts.
	
	\subsection{Axial}
	Multiply Eq.~\eqref{eq:axial_strong} by \(\delta u\), integrate by parts:
	\begin{equation}\label{eq:axial_weak}
		\int_0^L EA\,u'\,\delta u'\,dx 
		- \bigl[EA\,u'\,\delta u\bigr]_0^L = 0
		\;\Longrightarrow\;
		\int_0^L EA\,u'\,\delta u'\,dx = 0.
	\end{equation}
	
	\subsection{Bending}
	Multiply Eq.~\eqref{eq:transverse_strong} by \(\delta w\), integrate by parts twice:
	\begin{equation}\label{eq:bending_weak}
		\int_0^L EI\,w''\,\delta w''\,dx 
		- \bigl[EI\,w''\,\delta w' - (EI\,w'')'\,\delta w\bigr]_0^L = 0
		\;\Longrightarrow\;
		\int_0^L EI\,w''\,\delta w''\,dx = 0.
	\end{equation}
	
	\section{Finite‐Element Interpolations}
	\subsection{Axial (linear)}
	We approximate \(u(x)\) over each element of length \(L_e\) by
	\begin{equation}\label{eq:axial_interp}
		N^{(u)}(x)
		= \begin{pmatrix}1-\tfrac{x}{L_e}\\[4pt]\tfrac{x}{L_e}\end{pmatrix},
		\qquad
		u(x)=\bigl[N^{(u)}(x)\bigr]^T(u_1,u_2)^T.
	\end{equation}
	Because \(dN^{(u)}/dx=[-1/L_e,\;1/L_e]^T\) is constant, the axial strain is uniform, simplifying the stiffness integral.
	
	\subsection{Transverse (Hermite cubic)}
	To capture bending curvature with \(C^1\) continuity, we use four cubic polynomials:
	\begin{equation}\label{eq:cubic_interp}
		\begin{aligned}
			N_1(x)&=1-3(\tfrac{x}{L_e})^2+2(\tfrac{x}{L_e})^3,\\
			N_2(x)&=x[1-2(\tfrac{x}{L_e})+(\tfrac{x}{L_e})^2],\\
			N_3(x)&=3(\tfrac{x}{L_e})^2-2(\tfrac{x}{L_e})^3,\\
			N_4(x)&=x[-(\tfrac{x}{L_e})+(\tfrac{x}{L_e})^2].
		\end{aligned}
	\end{equation}
	Mapping \(x\to\xi=x/L_e\in[0,1]\) for integration:
	\begin{equation}\label{eq:map_interp}
		\begin{aligned}
			N_1(\xi)&=1-3\xi^2+2\xi^3,\quad
			N_2(\xi)=L_e\,\xi(1-2\xi+\xi^2),\\
			N_3(\xi)&=3\xi^2-2\xi^3,\quad
			N_4(\xi)=L_e\,\xi(-\xi+\xi^2).
		\end{aligned}
	\end{equation}
	Then \(w(x)=N^{(w)}(\xi)\,(w_1,\theta_1,w_2,\theta_2)^T\).
	
	\section{Element Stiffness Matrices}
	Substituting the above interpolations into the weak forms (\ref{eq:axial_weak})--(\ref{eq:bending_weak}) and evaluating the resulting integrals yields:
	
	\subsection{Axial}
	By inserting \(u(x)\) from \eqref{eq:axial_interp} into \eqref{eq:axial_weak} and noting the constant derivative,
	\begin{equation}\label{eq:ke_axial}
		k_e^{(\mathrm{axial})}
		= \int_0^{L_e}EA\,\frac{dN^{(u)}}{dx}\frac{dN^{(u)}}{dx}^Tdx
		= \frac{EA}{L_e}
		\begin{pmatrix}1 & -1\\ -1 & 1\end{pmatrix}.
	\end{equation}
	This 2×2 block arises directly from the axial strain energy.
	
	\subsection{Bending}
	From the bending weak form \eqref{eq:bending_weak}, define the curvature–shape vector in \(\xi\):
	\begin{equation}\label{eq:B_vector}
		B(\xi)
		= \frac{d^2N^{(w)}}{dx^2}
		= \frac{1}{L_e^2}[6\xi-6,\;3\xi-4,\;-6\xi+6,\;3\xi-2],
	\end{equation}
	then
	\begin{equation}\label{eq:ke_bending}
		k_e^{(\mathrm{bending})}
		= EI\int_0^{L_e}B^T B\,dx
		= \frac{EI}{L_e^3}
		\begin{pmatrix}
			12 & 6L_e & -12 & 6L_e\\
			6L_e & 4L_e^2 & -6L_e & 2L_e^2\\
			-12 & -6L_e & 12 & -6L_e\\
			6L_e & 2L_e^2 & -6L_e & 4L_e^2
		\end{pmatrix}.
	\end{equation}
	This 4×4 block represents the bending stiffness derived from curvature energy \(\tfrac12\int EI\,w''^2dx\).
	
	\subsection{Assembled stiffness}
	Finally, the full 6×6 element stiffness matrix is
	\begin{equation}\label{eq:ke_assembled}
		k_e
		= \begin{pmatrix}
			k_e^{(\mathrm{axial})} & \mathbf{0}\\
			\mathbf{0}            & k_e^{(\mathrm{bending})}
		\end{pmatrix}.
	\end{equation}
	
	\section{Element Mass Matrices}
	The element mass matrix comes from the beam’s kinetic energy,
	\[
	T
	= \tfrac12\int_0^{L_e}\rho A\,\dot u^2(x,t)\,dx
	\;+\;
	\tfrac12\int_0^{L_e}\rho A\,\dot w^2(x,t)\,dx.
	\]
	
	\subsection{Axial}
	We start with the axial kinetic energy
	\[
	T_{\mathrm{axial}}
	= \tfrac12\int_0^{L_e}\rho A\,\dot u^2\,dx.
	\]
	Substitute the linear interpolation \eqref{eq:axial_interp}:
	\[
	\dot u(x,t)
	= \bigl[N^{(u)}(x)\bigr]^T\,\dot{\mathbf{u}},
	\]
	which gives
	\[
	T_{\mathrm{axial}}
	= \tfrac12\,\dot{\mathbf{u}}^T
	\Bigl[\int_0^{L_e}\rho A\,N^{(u)}\,N^{(u)T}\,dx\Bigr]
	\,\dot{\mathbf{u}}
	= \tfrac12\,\dot{\mathbf{u}}^T\,m_e^{(\mathrm{axial})}\,\dot{\mathbf{u}}.
	\]
	Hence the consistent axial mass block is
	\begin{equation}\label{eq:me_axial}
		m_e^{(\mathrm{axial})}
		= \int_0^{L_e}\rho A\,N^{(u)}N^{(u)T}\,dx
		= \frac{\rho A L_e}{6}
		\begin{pmatrix}2 & 1\\ 1 & 2\end{pmatrix}.
	\end{equation}
	
	\subsection{Bending}
	Likewise, for bending
	\[
	T_{\mathrm{bend}}
	= \tfrac12\int_0^{L_e}\rho A\,\dot w^2\,dx.
	\]
	Using the Hermite‐cubic interpolation \eqref{eq:cubic_interp}:
	\[
	\dot w(x,t)
	= \bigl[N^{(w)}(\xi)\bigr]^T\,\dot{\mathbf{w}},
	\]
	we obtain
	\[
	T_{\mathrm{bend}}
	= \tfrac12\,\dot{\mathbf{w}}^T
	\Bigl[\int_0^{L_e}\rho A\,N^{(w)}(\xi)\,N^{(w)T}(\xi)\,dx\Bigr]
	\,\dot{\mathbf{w}}
	= \tfrac12\,\dot{\mathbf{w}}^T\,m_e^{(\mathrm{bending})}\,\dot{\mathbf{w}}.
	\]
	Thus the bending mass block is
	\begin{equation}\label{eq:me_bending}
		m_e^{(\mathrm{bending})}
		= \int_0^{L_e}\rho A\,N^{(w)}(\xi)\,N^{(w)T}(\xi)\,dx
		= \frac{\rho A L_e}{420}
		\begin{pmatrix}
			156 & 22L_e & 54 & -13L_e\\
			22L_e & 4L_e^2 & 13L_e & -3L_e^2\\
			54 & 13L_e & 156 & -22L_e\\
			-13L_e & -3L_e^2 & -22L_e & 4L_e^2
		\end{pmatrix}.
	\end{equation}
	
	\subsection{Assembled Mass}
	Putting axial and bending together gives the full element mass matrix:
	\begin{equation}\label{eq:me_assembled}
		m_e
		= \begin{pmatrix}
			m_e^{(\mathrm{axial})} & \mathbf{0}\\
			\mathbf{0}            & m_e^{(\mathrm{bending})}
		\end{pmatrix}.
	\end{equation}
	
	\bibliographystyle{elsarticle-num}
	\bibliography{references}
	
\end{document}
